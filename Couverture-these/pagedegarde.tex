%Pour diff\'{e}rentes universit\'{e}s il faurdra modifier ce d\'{e}but de document
%For different universities modify the beginning of this document to adapt the logos (logo)
\geometry{vmargin=4.0cm}
\logoecoledoc{./Couverture-these/MathSTIC/logo-mathSTIC} %logo ecole doctorale
\logoetablissement{./Couverture-these/MathSTIC/logo-etablissements/logoUR1}%logo etablissement (ici Rennes 1) 
\vspace{2.5cm}
%Indiquer l'\'{e}tablissement de d\'{e}livrance du diplome
%Provide the name of the institution that delivers the diploma, 
\unite{L'UNIVERSIT\'{E} DE RENNES 1}
%Si la th\`{e}se est en co-tutelle ou en d\'{e}livrance conjointe, mettre le logo des 2 \'{e}tablissements 
%in the case of "co-tutelle" provide both names (logos)
\ubl{Comue Université Bretagne Loire}
%Indiquer le num\'{e}ro d’accr\'{e}ditation de l’\'{e}cole doctorale de r\'{e}f\'{e}rence pour la th\`{e}se 
%indicate the certification number of l'ecole doctorale
\ecoledoc{\'{E}cole Doctorale N° 601}
%Indiquer le nom de l’\'{e}cole doctorale de r\'{e}f\'{e}rence pour la th\`{e}se 
%Indicate the name of l'ecole doctorale
\nomecole{Math\'{e}matiques et Sciences et Technologies \\ de l'Information et de la Communication}
%Inscrivez ici votre sp\'{e}cialit\'{e} (voir liste des sp\'{e}cialit\'{e}s sur le site de votre \'{e}cole doctorale)
%Indicate the domain (see list of domains in your ecole doctorale)
\spec{Sp\'{e}cialit\'{e} : \textit{(voir liste des sp\'{e}cialit\'{e}s)}}
\usepackage{eso-pic}


\begin{document}

% PDF document Tags
\hypersetup{
    pdftitle={title},    % title
    pdfauthor={author},     % author
    pdfsubject={Subject},   % subject of the document
    pdfcreator={creator},   % creator of the document
    pdfproducer={producer}, % producer of the document
    pdfkeywords={Green Networking} {Mobile Cloud} {Network Coding} {Energy} % 
}

% La page de garde est en français
% The front cover is in French
\selectlanguage{french}

%background image of the front cover
\AddToShipoutPicture*{%
    \put(0,0){%
    \parbox[b][42.5cm]{\paperwidth}{%
        \vfill
        \includegraphics[width=\paperwidth,keepaspectratio]{./Couverture-these/MathSTIC/image-fond-MATHSTIC-gardeO.png}% image de fond UR1
        \vfill
}}}
%Attention : le pr\'{e}nom doit être en minuscules (Jean) et le NOM en majuscules (BRITTEF) 
%Attention : the first name in small letters and the name in Capital letters 
\author{« Pr\'{e}nom NOM »}
% Donner le titre complet de la th\`{e}se, \'{e}ventuellement le sous titre, si n\'{e}cessaire sur plusieurs lignes 
%Give the complete title of the thesis, if necessary on several lines
\title{« Titre de la th\`{e}se »}
\lesoustitre{« Sous-titre de la th\`{e}se »}
%Indiquer la date et le lieu de soutenance de la th\`{e}se 
%indicates the date and the place of the defense 
\date{« date »}
\lieu{« Lieu »}
%Indiquer le nom du (ou des) laboratoire (s) dans le(s)quel(s) le travail de th\`{e}se a \'{e}t\'{e} effectu\'{e}, indiquer aussi si souhait\'{e} le nom de la (les) facult\'{e}(s) (UFR, \'{e}cole(s), Institut(s), Centre(s)...), son (leurs) adresse(s)... 
%Indicates the name (or names) of research laboratories where the work has been done as well as (if desired) the names of faculties (UFR, Schools, institution...
\uniterecherche{Unit\'{e} de recherche : }
%Indiquer le Numero de th\`{e}se, si cela est opportun, ou faire disparaitre cet item de la couverture 
%Indicate the number of the thesis if there is one.
\numthese{Th\`{e}se N° :  }
%Indiquer le Pr\'{e}nom en minuscules et le Nom en majuscules, le titre de la personne et l’\'{e}tablissement dans lequel il effectue sa recherche  
%Indicates the first name on small letters and the Names on capital letters, the person's title and the institution where he/she belongs to.
%Exemples :  Examples :
%%%- Professeur, Universit\'{e} d’Angers 
%%%- Chercheur, CNRS, \'{e}cole Centrale de Nantes 
%%%-  Professeur d’universit\'{e} – Praticien Hospitalier, Universit\'{e} Paris V  
%%%-  Maitre de conf\'{e}rences, Oniris 
%%%- Charg\'{e} de recherche, INSERM, HDR, Universit\'{e} de Tours  
 %S’il n’y a pas de co-direction, faire disparaitre cet item de la couverture  
 %In there is no co-director, remove the item from the cover


\jury{
{\normalsize \textbf{Rapporteurs avant soutenance :}}\\ \newline
\footnotesize
\begin{tabular}{@{}ll}
Pr\'{e}nom Nom & Fonction et \'{e}tablissement d'exercice \\
Pr\'{e}nom Nom & Fonction et \'{e}tablissement d'exercice \\
Pr\'{e}nom Nom & Fonction et \'{e}tablissement d'exercice \\
\end{tabular}

\vspace{\baselineskip}
{\normalsize \textbf{Composition du Jury :}}\\
{\fontsize{9.5}{11}\selectfont {\textcolor{red}{\textit{Attention, en cas d’absence d’un des membres du Jury le jour de la soutenance, la composition du jury doit être revue pour s’assurer qu’elle est conforme et devra être répercutée sur la couverture de thèse}}}}\\ \newline
\footnotesize
\begin{tabular}{@{}lll}

Pr\'{e}sident :        & Pr\'{e}nom Nom & Fonction et \'{e}tablissement d'exercice \textit{(à préciser après la soutenance)} \\
Examinateurs :         & Pr\'{e}nom Nom & Fonction et \'{e}tablissement d'exercice \\
                       & Pr\'{e}nom Nom & Fonction et \'{e}tablissement d'exercice \\
                       & Pr\'{e}nom Nom & Fonction et \'{e}tablissement d'exercice \\
                       & Pr\'{e}nom Nom & Fonction et \'{e}tablissement d'exercice \\
Dir. de th\`{e}se :    & Pr\'{e}nom Nom & Fonction et \'{e}tablissement d'exercice \\
Co-dir. de th\`{e}se : & Pr\'{e}nom Nom & Fonction et \'{e}tablissement d'exercice \textit{(si pertinent)} \\
\end{tabular}

\vspace{\baselineskip}
{\normalsize \textbf{Invit\'{e}(s) :}}\\
\footnotesize
\begin{tabular}{@{}ll}
Pr\'{e}nom Nom & Fonction et \'{e}tablissement d'exercice \\
\end{tabular}
}


\maketitle
